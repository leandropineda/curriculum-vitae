% !TEX TS-program = xelatex
% !TEX encoding = UTF-8 Unicode
% -*- coding: UTF-8; -*-
% vim: set fenc=utf-8

%%%%%%%%%%%%%%%%%%%%%%%%%%%%%%%%%%%%%%%%%%%%%%%%%%%%%%%%%%%%%%%%%
%% SIMPLE-RESUME-CV
%% <https://github.com/zachscrivena/simple-resume-cv>
%% This is free and unencumbered software released into the
%% public domain; see <http://unlicense.org> for details.
%%%%%%%%%%%%%%%%%%%%%%%%%%%%%%%%%%%%%%%%%%%%%%%%%%%%%%%%%%%%%%%%%

%%%%%%%%%%%%%%%%%%%%%%%%%%%%%%%%%%%%%%%%%%%%%%%%%%%%%%%%%%%%%%%%%
%% INSTRUCTIONS FOR COMPILING THIS DOCUMENT ("CV.tex")
%% TeX ---(XeLaTeX)---> PDF:
%%
%% Method 1: Use latexmk for fully automated document generation:
%%   latexmk -xelatex "CV.tex"
%%   (add the -pvc switch to automatically recompile on changes)
%%
%% Method 2: Use XeLaTeX directly:
%%   xelatex "CV.tex"
%%   (run multiple times to resolve cross-references if needed)
%%%%%%%%%%%%%%%%%%%%%%%%%%%%%%%%%%%%%%%%%%%%%%%%%%%%%%%%%%%%%%%%%

\documentclass[letterpaper,MMMyyyy,nonstop]{simpleresumecv}
% Class options:
% a4paper, letterpaper, draft, nonstop
% MMMyyyy, ddMMMyyyy, MMMMyyyy, ddMMMMyyyy, yyyyMMdd, yyyyMM, yyyy

%%%%%%%%%%%%%%%%%%%%%%%%%%%%%%%%%%%%%%%%%%%%%%%%%%%%%%%%%%%%%%%%%
%% PREAMBLE.
%%%%%%%%%%%%%%%%%%%%%%%%%%%%%%%%%%%%%%%%%%%%%%%%%%%%%%%%%%%%%%%%%

% CV Info (to be customized).
\newcommand{\CVAuthor}{Leandro Pineda}
\newcommand{\CVTitle}{Curriculum Vitae}
\newcommand{\CVNote}{Last update {\Timestamp}}
\newcommand{\CVWebpage}{}

% PDF settings and properties.
\hypersetup{
pdftitle={\CVTitle},
pdfauthor={\CVAuthor},
pdfsubject={\CVWebpage},
pdfcreator={XeLaTeX},
pdfproducer={},
pdfkeywords={},
pdfpagemode={},
unicode=true,
bookmarks=true,
bookmarksopen=true,
pdfstartview=FitH,
pdfpagelayout=OneColumn,
pdfpagemode=UseOutlines,
colorlinks=true,       % false: boxed links; true: colored links
linkcolor=red,          % color of internal links (change box color with linkbordercolor)
citecolor=green,        % color of links to bibliography
filecolor=magenta,      % color of file links
urlcolor=cyan           % color of external links
}

% Shorthand.
\newcommand{\CodeCommand}[1]{\mbox{\textbf{\textbackslash{#1}}}}

%%%%%%%%%%%%%%%%%%%%%%%%%%%%%%%%%%%%%%%%%%%%%%%%%%%%%%%%%%%%%%%%%
%% ACTUAL DOCUMENT.
%%%%%%%%%%%%%%%%%%%%%%%%%%%%%%%%%%%%%%%%%%%%%%%%%%%%%%%%%%%%%%%%%

\begin{document}

%%%%%%%%%%%%%%%
% TITLE BLOCK %
%%%%%%%%%%%%%%%

\title{\CVAuthor}
\vspace{.5cm}
\begin{subtitle}
Santa Rosa 866, 9F, Córdoba, Argentina\, \SubBulletSymbol\, 31 years old
\par
\hspace{-4cm}\href{mailto:leandropineda.lp@gmail.com}
{leandropineda.lp@gmail.com}
\,\SubBulletSymbol\,
+54\,342\,4-794-999
\end{subtitle}
 
\hspace{14.5cm}\smash{\includegraphics[width=3.5cm]{photo.jpg}}
\begin{body}

%%%%%%%%%%%%%%%
%% EDUCATION %%
%%%%%%%%%%%%%%%

\section
{Education}
{Education}
{PDF:Education}

\textbf{Universidad Nacional del Litoral},
Santa Fe, Santa Fe, Argentina
\hfill
\DatestampY{2009} --
\DatestampY{2018}

\BulletItem Facultad de Ingeniería y Ciencias Hídricas
\begin{detail}
	\SubBulletItem
	Engineer's Degree: \href{http://fich.unl.edu.ar/planificaciones/carrera.php?id=3}{Computer Engineering}.
	\SubBulletItem
	G.P.A.: 8.10.
\end{detail}
\BulletItem	Thesis: \href{https://github.com/leandropineda/sketch-ws}{Designing a computer network anomaly detection system}.
\begin{detail}
	\SubBulletItem
	Goal: design and build a distributed system for anomalous behaviour detection, analysing network traffic and log files content from different application services like apache and mysql. This will be accomplished using \textit{data streaming} techniques for processing data from different sources in a real-time fashion.
\end{detail}

\BigGap
\textbf{E.E.T. Nº 478 "Dr. Nicolás Avellaneda"},
Santa Fe, Santa Fe, Argentina
\hfill
\DatestampY{2005} --
\DatestampY{2008}
\BulletItem Educación Polimodal
\begin{detail}
	\SubBulletItem
	Bachelor Degree: Computer Technician.
	\SubBulletItem
	G.P.A.: 8.05.
\end{detail}

%\href{http://www.example.com/my-university}{\textbf{Escuela Industrial Superior}},
%\BigGap
%\textbf{Escuela Industrial Superior},
%Santa Fe, Santa Fe, Argentina
%\hfill
%\DatestampY{2002} --
%\DatestampY{2004}
%\BulletItem
%Educación General Básica.
%
%\BigGap
%\textbf{Escuela Nº 331 “Alte. Guillermo Brown”},
%Santa Fe, Santa Fe, Argentina
%\hfill
%\DatestampY{2001}{}
%\BulletItem
%Educación General Básica.
%
%\BigGap
%\textbf{Colegio Don Bosco},
%Bahia Blanca, Buenos Aires, Argentina
%\hfill
%\DatestampY{1995} --
%\DatestampY{2000}
%\BulletItem
%Educación General Básica.

%%%%%%%%%%%%%%%%%%%%%%%%%
%% RESEARCH EXPERIENCE %%
%%%%%%%%%%%%%%%%%%%%%%%%%

%\section
%{Research Experience}
%{Research Experience}
%{PDF:ResearchExperience}
%
%\href{http://www.example.com/my-institute}
%{\textbf{Institute for Advanced Research}},
%Science College
%
%\GapNoBreak
%\BulletItem
%Undergraduate Research Student, Science Department
%\hfill
%\DatestampYMD{2004}{05}{15} --
%\DatestampYMD{2005}{05}{15}
%\begin{detail}
%\SubBulletItem
%Project:
%Investigations on the Use of Lasers to Measure Climate Change
%\SubBulletItem
%Supervisors:
%Prof.~Jane~Citizen and
%Dr~Ann~Yone
%\SubBulletItem
%Focus:
%Climate change, lasers, statistical analysis, data analytics.
%\end{detail}

%%%%%%%%%%%%%%%%%%
%% PUBLICATIONS %%
%%%%%%%%%%%%%%%%%%

%\section
%{Publications}
%{Publications}
%{PDF:Publications}
%
%\subsection
%{Journals}
%{Journals}
%{PDF:Journals}
%
%\GapNoBreak
%\NumberedItem{[11]}
%\href{http://www.example.com/my-paper-doi-5}
%{\underline{J.~Doe}, J.~Citizen, and A.~Yone,
%``On lasers and climate change,''
%\textit{Journal of Science},
%vol.~89,
%no.~2,
%pp.~4123--4133,
%\DatestampYM{2008}{02}.}
%
%% Note the use of {\CharSpace} for aligning shorter numbers.
%\Gap
%\NumberedItem{{\CharSpace}[1]}
%\href{http://www.example.com/my-paper-doi-4}
%{\underline{J.~Doe} and J.~Citizen,
%``Measuring the extent of climate change,''
%\textit{Global Scientific Journal},
%vol.~12,
%no.~4,
%pp.~330--352,
%\DatestampYM{2006}{12}.}
%
%\BigGap
%\subsection
%{Conferences}
%{Conferences}
%{PDF:Conferences}
%
%\GapNoBreak
%\NumberedItem{[11]}
%\href{http://www.example.com/my-paper-doi-3}
%{\underline{J.~Doe}, J.~Citizen, and A.~Yone,
%``On lasers and climate change,''
%in \textit{Proceedings of the Laser Symposium},
%Las Vegas, Nevada, USA,
%\DatestampYM{2007}{01}.}
%
%\Gap
%\NumberedItem{[10]}
%\href{http://www.example.com/my-paper-doi-2}
%{A.~Yone and \underline{J.~Doe},
%``Climate change and general relativity,''
%in \textit{Proceedings of the International Astronomical Conference},
%Sydney, Australia,
%\DatestampYM{2006}{8}.}
%
%% Note the use of {\CharSpace} for aligning shorter numbers.
%\Gap
%\NumberedItem{{\CharSpace}[1]}
%\href{http://www.example.com/my-paper-doi-1}
%{\underline{J.~Doe} and J.~Citizen,
%``Measuring the extent of climate change,''
%in \textit{Proceedings of the International Climate Change Conference},
%London, UK,
%\DatestampYM{2005}{11}.}

%%%%%%%%%%%%%%%%%%%%%%%%%%%%%%%%%%%%%%%%%%%%
%% PROFESSIONAL AFFILIATIONS & ACTIVITIES %%
%%%%%%%%%%%%%%%%%%%%%%%%%%%%%%%%%%%%%%%%%%%%

%\section
%{Professional Affiliations\newline
%\& Activities}
%{Professional Affiliations \& Activities}
%{PDF:ProfessionalAffiliationsActivities}
%
%\href{http://www.example.com/my-society}
%{\textbf{Society of Professional Earth Scientists}},
%New York, USA
%
%\GapNoBreak
%\BulletItem
%Member
%\hfill
%\DatestampY{2009} --
%Present

%%%%%%%%%%%%%%%%%%%%%%%
%% CAMPUS ACTIVITIES %%
%%%%%%%%%%%%%%%%%%%%%%%

%\section
%{Campus Activities}
%{Campus Activities}
%{PDF:CampusActivities}
%
%\href{http://www.example.com/my-club}
%{\textbf{First Volunteers Club}},
%First American University
%
%\GapNoBreak
%\BulletItem
%President
%\hfill
%\DatestampYMD{2006}{08}{15} --
%\DatestampYMD{2007}{08}{15}
%\begin{detail}
%\SubBulletItem
%Lorem ipsum dolor sit amet, consectetur adipiscing elit.
%\SubBulletItem
%Curabitur vitae laoreet velit, vel ultricies est. Nam nec elit ac ante facilisis ultrices.
%\SubBulletItem
%Integer sit amet turpis dolor. Lorem ipsum dolor sit amet, consectetur adipiscing elit. Nunc at orci eu leo vulputate finibus sed et sem.
%\SubBulletItem
%Suspendisse volutpat sapien et mi cursus, gravida ornare mauris sollicitudin.
%\end{detail}

%%%%%%%%%%%%%%%%%%%%%%%%%%%
%% OTHER WORK EXPERIENCE %%
%%%%%%%%%%%%%%%%%%%%%%%%%%%

\section
{Professional \newline
	Experience}
{ProfessionalExperience}
{PDF:ProfessionalExperience}

\href{https://www.binbash.com.ar/}
{\textbf{BinBash - DevOps Cloud Services.}},
Córdoba, Córdoba, Argentina

\GapNoBreak
\BulletItem
\textbf{Role}: DevOps Engineer.
\hfill
\DatestampYM{2018}{11} --
\DatestampYM{2019}{08}
\begin{detail}
	\SubBulletItem
	Maintaining and building Infrastructure as Code.
	\SubBulletItem
	DevSecOps pipelines \& Automation.
	\SubBulletItem
	Kubernetes cluster administration.
\end{detail}

\BulletItem Short description: Assigned to operate and maintain customer infrastructure on AWS environments (kubernetes), building telemetry dashboards and alerting capabilities, and automating security assessments. Heavily focused on SecOps and automation using Jenkins, Python scripting and AWS services.


\GapNoBreak
\href{http://www.mcafee.com}
{\textbf{McAfee, Inc.}},
Córdoba, Córdoba, Argentina

\GapNoBreak
\BulletItem
\textbf{Role}: Software Engineer - Tech Lead
\hfill
\DatestampYM{2017}{01} --
Today
\begin{detail}
	\SubBulletItem
	Python developer.
	\SubBulletItem
	Automation.
\end{detail}

\BulletItem Short description: We are building a cloud platform solution to assist Security Operation Centers for Enterprise companies. My main role is to develop python code to gather and process massive amounts of data from different sources to present them to security analysts in a meaningful way. Currently migrating and leading code base migration to Python 3.

\GapNoBreak
\href{http://www.sinc.unl.edu.ar}
{\textbf{Research Institute sinc(i)}},
Santa Fe, Santa Fe, Argentina

\GapNoBreak
\BulletItem
\textbf{Role}: IT Support Technician
\hfill
\DatestampYM{2014}{08} --
\DatestampYM{2016}{12}
\begin{detail}
\SubBulletItem
GNU/Linux server administration.
\SubBulletItem
Network maintenance.
\SubBulletItem
DevOps at project: \href{https://bitbucket.org/sinc-lab/webdemobuilder-docker}{https://bitbucket.org/sinc-lab/webdemobuilder-docker}
\end{detail}

%%%%%%%%%%%%%%%%%%
%% CAPACITACION %%
%%%%%%%%%%%%%%%%%%

\section
{Courses}
{Courses}
{PDF:Courses}


%%%%%%%%%%%%%%%%%%%%%%%%%%%%%%%%%%%%%%%%%%%%%%%%%%%%%%%%%%%%%%%%%%%%%%%%%%%%%%%%%%%%%%%%%%%%%%%
\href{https://www.ets.org/toeic}{\textbf{Test of English for International Communication (TOEIC)}}
\hfill
\DatestampYM{2018}{10}

\begin{detail}
	\SubBulletItem 
	Accepted and trusted by 14,000+ organizations in more than 160 countries, the TOEIC® tests assess your English-language proficiency across all four language skills needed to succeed in the global workplace - listening, reading, speaking and writing.
\end{detail}

\BigGap
%%%%%%%%%%%%%%%%%%%%%%%%%%%%%%%%%%%%%%%%%%%%%%%%%%%%%%%%%%%%%%%%%%%%%%%%%%%%%%%%%%%%%%%%%%%%%%%
\href{https://www.sans.org/course/advanced-incident-response-threat-hunting-training}{\textbf{FOR508: Advanced Digital Forensics, Incident Response, and Threat Hunting}}
\hfill
\DatestampYM{2018}{02}

\BulletItem FOR508: Advanced Digital Forensics, Incident Response, and Threat Hunting. SANS Institute.
\begin{detail}
	\SubBulletItem 
	Topics: detect how and when a breach occurred, identify compromised and affected systems, determine what attackers took or changed, contain and remediate incidents, develop key sources of threat intelligence, hunt down additional breaches using knowledge of the adversary.	
	\SubBulletItem
	Certificate: \href{https://www.youracclaim.com/badges/3b26b130-d265-4d5e-9d8f-cd6fdc95c925/public\_url}{https://www.youracclaim.com/badges/3b26b130-d265-4d5e-9d8f-cd6fdc95c925/public\_url}
\end{detail}
Duration: 6 days.

\BigGap
%%%%%%%%%%%%%%%%%%%%%%%%%%%%%%%%%%%%%%%%%%%%%%%%%%%%%%%%%%%%%%%%%%%%%%%%%%%%%%%%%%%%%%%%%%%%%%%
\textbf{LAN under MikroTik: Layer 1 \& 2}
\hfill
\DatestampYM{2016}{10}

\BulletItem Goal: Introduce common layer 1 and 2 protocols, and their configuration using Mikrotik.

\begin{detail}
	\SubBulletItem 
	Topics: Ethernet configuration. Speed: Full Duplex/Half Duplex, statistics. Understanding Torch. Bridge mode.  L2 filters. L2 NAT. STP y RST. Bridge ethernet-WiFi. VLAN. Cloud routers. IVL. CVS. Mesh networks. HWMPlus networks. L2 point-to-point connections. PPP and L2TP. Profiles. Client/Server configuration. PPPoE protocol.
	\SubBulletItem Teaching: Pablo Roa.
\end{detail}
Duration: 32 hours \SubBulletSymbol\, Santa Fe, Argentina.

\BigGap
%%%%%%%%%%%%%%%%%%%%%%%%%%%%%%%%%%%%%%%%%%%%%%%%%%%%%%%%%%%%%%%%%%%%%%%%%%%%%%%%%%%%%%%%%%%%%%%
\href{https://www.coursera.org/learn/progfun1}{\textbf{Functional Programming Principles in Scala}}
\hfill
\DatestampYM{2016}{07}

\BulletItem École Polytechnique Fédérale de Lausanne at \href{https://www.coursera.org/}{www.coursera.org}.
\begin{detail}
	\SubBulletItem
	Topics: Functions \& Evaluation. Higher Order Functions. Data and Abstraction. Types and Pattern Matching. Lists. Collections
	\SubBulletItem
	Certificate: \href{https://www.coursera.org/account/accomplishments/records/H6WQNR37R6Y5}
	{https://www.coursera.org/account/accomplishments/records/H6WQNR37R6Y5}
	
\end{detail}


\BigGap
%%%%%%%%%%%%%%%%%%%%%%%%%%%%%%%%%%%%%%%%%%%%%%%%%%%%%%%%%%%%%%%%%%%%%%%%%%%%%%%%%%%%%%%%%%%%%%%
\href{https://www.edx.org/course/introduction-to-computer-science-and-programming-using-python}{\textbf{6.00.1x: Introduction to Computer Science and Programming Using Python}}
\hfill
\DatestampYM{2016}{03}

\BulletItem Massachusetts Institute of Technology at \href{http://www.edx.org/}{www.edx.org}.
\begin{detail}
	\SubBulletItem
	Topics: Introduction. Core elements of programs. Simple algorithms. Functions. Recursion. Objects. Debugging. Assertions and Exceptions. Efficiency and orders of growth. Memory and search. Classes. Object Oriented Programming	and Inheritance. Trees.
	\SubBulletItem
	Certificate: \href{https://courses.edx.org/certificates/c8d7817c072644b191489164127b815d}
	{https://courses.edx.org/certificates/c8d7817c072644b191489164127b815d}
	
\end{detail}


\BigGap
\textbf{Network Management using Mikrotik}
\hfill
\DatestampYM{2015}{12}

\BulletItem Goals: Learn about Routerboard hardware platform. Applications and performance. Learn to manage a Mikrotik node in different modes. Compare specifications and hardware requirements.
\begin{detail}
	\SubBulletItem 
	Topics: Different hardware platforms. Routerboard types: performance and applications. Licenses. Routerboard and PC. Routerboard BIOS management. Booting from FLASH NAND. Management and remote access. RS232 access. Command line interface. Password reset. Software upgrade. Mikrotik information repository. Common failures and problems.
	\SubBulletItem Teaching: Pablo Roa.
\end{detail}
Duration: 20 hours \SubBulletSymbol\, Santa Fe, Argentina.


%%%%%%%%%%%%%%%%%%%%%%%%
%% CONFERENCES AND SEMINARS %%
%%%%%%%%%%%%%%%%%%%%%%%%

\section{Conferences\newline 
	and Seminars}
{ConferencesSeminars}
{PDF:ConferencesSeminars}

%%%%%%%%%%%%%%%%%%%%%%%%%%%%%%%%%%%%%%%%%%%%%%%%%%%%%%%%%%%%%%%%%%%%%%%%%%%%%%%%%%%%%%%%%%%%%%%
\href{https://reinforce.awsevents.com/}{\textbf{AWS re:Inforce}}
\hfill
\DatestampYM{2019}{06}

\BulletItem Boston Convention and Exhibition Center.
\begin{detail}
	\SubBulletItem
	AWS re:Inforce is a learning conference focused on cloud security, identity, and compliance. The event includes hundreds of technical sessions, a keynote featuring AWS security leadership, and access to cloud security experts in the Security Learning Hub.

\end{detail}
Duration: 3 days \SubBulletSymbol\, Boston, Massachusetts.

\BigGap

%%%%%%%%%%%%%%%%%%%%%%%%%%%%%%%%%%%%%%%%%%%%%%%%%%%%%%%%%%%%%%%%%%%%%%%%%%%%%%%%%%%%%%%%%%%%%%%
\href{https://www.rsaconference.com/events/us18}{\textbf{RSA Conference}}
\hfill
\DatestampYM{2018}{04}

\BulletItem Moscone Center.
\begin{detail}
	\SubBulletItem
	RSA Conference conducts information security events around the globe that connect you to industry leaders and highly relevant information.
\end{detail}
Duration: 3 days \SubBulletSymbol\, San Francisco, California.

\BigGap

%%%%%%%%%%%%%%%%%%%%%%%%%%%%%%%%%%%%%%%%%%%%%%%%%%%%%%%%%%%%%%%%%%%%%%%%%%%%%%%%%%%%%%%%%%%%%%%
\href{https://www.ekoparty.org/acerca-ekoparty.php}{\textbf{Ekoparty Security Conference}}
\hfill
\DatestampYM{2017}{09}

\BulletItem Ciudad cultural KONEX.
\begin{detail}
	\SubBulletItem
	The eko is held annually in the Autonomous City of Buenos Aires. In this event, attendees, guests, specialists and referrers on this field from all around the world, have the opportunity to get involved with the latest technological innovations, vulnerabilities and tools. The talks are translated simultaneously and can be heard in Spanish and English
\end{detail}
Duration: 3 days \SubBulletSymbol\, Buenos Aires, Argentina.

\BigGap

%%%%%%%%%%%%%%%%%%%%%%%%%%%%%%%%%%%%%%%%%%%%%%%%%%%%%%%%%%%%%%%%%%%%%%%%%%%%%%%%%%%%%%%%%%%%%%%
\href{http://45jaiio.sadio.org.ar/}{\textbf{45º JAIIO - Jornadas Argentinas de Informática}}
\hfill
\DatestampYM{2016}{09}

\BulletItem Sociedad Argentina de Informática – Centro Cultural Borges at UNTREF.
\begin{detail}
	\SubBulletItem
	Each JAIIOs is organized as a set of one or two days long separated symposia, each one dedicated to a specific area, in order to allow interaction between their attendants. Parallel sessions are held where papers from annual reports and research results are discussed, as well as conferences and meetings with Argentinian and foreign professionals.
\end{detail}
Duration: 3 days \SubBulletSymbol\, Buenos Aires, Argentina.

\BigGap

%%%%%%%%%%%%%%%%%%%%%%%%%%%%%%%%%%%%%%%%%%%%%%%%%%%%%%%%%%%%%%%%%%%%%%%%%%%%%%%%%%%%%%%%%%%%%%%
\textbf{Introduction to Application Security - Microsoft Research}
\hfill
\DatestampYM{2016}{07}

\BulletItem Escuela de Ciencias Informáticas – Departamento de Computación, UBA.
\begin{detail}
	\SubBulletItem
	This course will introduce students to adversarial mindset, i.e., how to think like an attacker, and will include an overview of the most common types of vulnerabilities in use in exploitation today, including buffer overruns and the most common web vulnerabilities such as cross-site scripting and SQL injection.
	\SubBulletItem
	Topics: Introduction to the adversarial mindset. Memory safety and buffer overruns. Web application security. Privacy. Tools for static and runtime analysis. Hacking for fun and profit: case studies: cars, routers, and other devices.
	\SubBulletItem
	Teaching: Ben Livshits.
	\SubBulletItem
	Grade: 8 (eight).
\end{detail}
Duration: 5 days \SubBulletSymbol\, Buenos Aires, Argentina.

\BigGap

%%%%%%%%%%%%%%%%%%%%%%%%%%%%%%%%%%%%%%%%%%%%%%%%%%%%%%%%%%%%%%%%%%%%%%%%%%%%%%%%%%%%%%%%%%%%%%%
\textbf{Cyber-security principles for enterprise environments - Intel Security}
\hfill
\DatestampYM{2016}{07}

\BulletItem Escuela de Ciencias Informáticas – Departamento de Computación, UBA.
\begin{detail}
	\SubBulletItem
	Goal: present how the security of an enterprise is structured to ensure their protection. Also are discussed some known tactics, techniques and procedures (TTPs) that are being used in cyber-attacks.
	\SubBulletItem
	Topics: Anatomy of an attack and the \textit{Cyber Kill Chain}. Network security and enterprise infrastructure. Defence in depth. Protecting enterprise networks. Protecting enterprise \textit{endpoints}.
	\SubBulletItem
	Teaching: Matías Cuenca-Acuna, Leonardo Frittelli, Marcelo Lorenzati,  María  Emilia  Torino and Gustavo Yaguez.
	\SubBulletItem
	Grade: 10 (ten).
\end{detail}
Duration: 5 days \SubBulletSymbol\, Buenos Aires, Argentina.

\BigGap
%%%%%%%%%%%%%%%%%%%%%%%%%%%%%%%%%%%%%%%%%%%%%%%%%%%%%%%%%%%%%%%%%%%%%%%%%%%%%%%%%%%%%%%%%%%%%%%
\textbf{VI Iberoamerican Congress of Researchers and Teachers of Law and Informatics}
\hfill
\DatestampYM{2016}{05}

\BulletItem Facultad de Ingeniería y Ciencias Hídricas – Universidad Nacional del Litoral.
\begin{detail}
	\SubBulletItem
	The Iberoamerican Congress of Researchers and Teachers of Law and Informatics, is a meeting that convene teachers and researchers from both Law and Informatics areas, with the intention to create an area that allows sharing and developing further investigation, to generate ties of cooperation in order to deepen the knowledge from debate and exchange of ideas.
\end{detail}
Duration: 3 days \SubBulletSymbol\, Santa Fe, Argentina.

\BigGap
%%%%%%%%%%%%%%%%%%%%%%%%%%%%%%%%%%%%%%%%%%%%%%%%%%%%%%%%%%%%%%%%%%%%%%%%%%%%%%%%%%%%%%%%%%%%%%%
\textbf{The role of vision in entrepreneurship}
\hfill
\DatestampYM{2015}{02}

\BulletItem Secretaría de Vinculación Tecnológica y Desarrollo Productivo – Universidad Nacional del Litoral.
\begin{detail}
	\SubBulletItem
	Goal: Identify the importance of both mission and vision, and also how important is for a successful enterprise, providing the main techniques and tools that facilitate its materialization.
	\SubBulletItem
	Topics: Characteristics of the entrepreneur. The mission of the entrepreneurship. The vision of the future of the entrepreneur. Carrying forward the mission and vision: the Business Plan.
	\SubBulletItem
	Teaching: Gustavo Miazzi and Hugo Amante. 
\end{detail}
Duration: 6 hours \SubBulletSymbol\, Santa Fe, Argentina.

%%%%%%%%%%%%%%%
%% LANGUAGES %%
%%%%%%%%%%%%%%%

\section
{Languages}
{Languages}
{PDF:Languages}

\BulletItem
English: Fluent.
\GapNoBreak
\BulletItem
Spanish: Native.
\GapNoBreak
\BulletItem
Dual Citizenship: Argentine/Spanish

%%%%%%%%%%%%
%% SKILLS %%
%%%%%%%%%%%%

\section
{Skills}
{Skills}
{PDF:Skills}

\textbf{Programming}
\BulletItem Python \SubBulletSymbol\, C++ \SubBulletSymbol\, Java \SubBulletSymbol\, Bash

\textbf{Tools}
\BulletItem
Technologies / Framework: Docker \SubBulletSymbol\, Flask \SubBulletSymbol\, Redis \SubBulletSymbol\, Kubernetes \SubBulletSymbol\, Kafka \SubBulletSymbol\, AWS
\SubBulletSymbol\, Terraform \SubBulletSymbol\, Ansible \SubBulletSymbol\, Grafana \BulletItem
Other Tools: JIRA \SubBulletSymbol\, TeamCity \SubBulletSymbol\, Jenkins

%%%%%%%%%%%
%% LINKS %%
%%%%%%%%%%%

\section
{Links}
{Links}
{PDF:Links}

\BulletItem
\href{https://github.com/leandropineda}{github.com/leandropineda}
\BulletItem
\href{https://www.linkedin.com/in/leandropineda-lp}{linkedin.com/in/leandropineda-lp}

%%%%%%%%%%%%%%%
%% INTERESTS %%
%%%%%%%%%%%%%%%

\section
{Interests}
{Interests}
{PDF:Interests}

Systems Security, Purple Teaming, Enterprise Security, Microservices, Automation, Digital Electronics \& Microcontrollers.

%%%%%%%%%%%%%%%
%% AFICIONES %%
%%%%%%%%%%%%%%%

%\section
%{Hobbies \newline
%	and Crafts}
%{HobbiesCrafts}
%{PDF:HobbiesCrafts}
%
%Running, Crossfit.


%%%%%%%%%%%%%%%%%
%%% REFERENCES %%
%%%%%%%%%%%%%%%%%
%
%\section
%{References}
%{References}
%{PDF:References}
%
%\BulletItem
%\textbf{Professor Jonathan Public}
%\newline
%Professor of Geology and Mechanical Engineering
%\newline
%First American University
%\newline
%1000 First Avenue, Springfield, Massachusetts 22222, USA
%\newline
%\href{mailto:jonathanpublic@example.com}
%{jonathanpublic@example.com}
%\,\SubBulletSymbol\,
%+1\,(555)\,222-2222
%
%\BigGap
%\BulletItem
%\textbf{Dr Alice Bob Carol}
%\newline
%Director, Research \& Development
%\newline
%Alpha Engineering Firm
%\newline
%20 North Street, Oakland, Ohio 33333, USA
%\newline
%\href{mailto:alicebobcarol@example.com}
%{alicebobcarol@example.com}
%\,\SubBulletSymbol\,
%+1\,(555)\,333-3333
%
%%%%%%%%%%%%%%%%%%%%%%%%%%%%%%%%%%
%%% SECTION WITH USAGE EXAMPLES %%
%%%%%%%%%%%%%%%%%%%%%%%%%%%%%%%%%%
%
%\section
%{Section\newline
%With\newline
%Usage\newline
%Examples}
%{Section With Usage Examples (For PDF Bookmark)}
%{PDF:SectionWithUsageExamples:ForPDFLink}
%
%\subsection
%{This is a Subsection}
%{This is a Subsection}
%{PDF:ThisIsASubSection}
%
%\GapNoBreak
%\BulletItem
%Use \CodeCommand{section} and \CodeCommand{subsection} to create sections and subsections.
%These will appear in the PDF bookmarks too.
%
%\GapNoBreak
%\BulletItem
%This is the second \CodeCommand{BulletItem}.
%Long items are automatically indented.
%Lorem ipsum dolor sit amet, consectetur adipiscing elit.
%Sed sed aliquam massa.
%\begin{detail}
%\SubBulletItem
%This is a \CodeCommand{SubBulletItem}.
%Long items are automatically indented.
%Lorem ipsum dolor sit amet, consectetur adipiscing elit.
%Sed sed aliquam massa.
%Aliquam dignissim mi non enim feugiat elementum.
%Donec sit amet turpis ac velit ultrices volutpat.
%Aliquam vitae elit massa.
%\SubBulletItem
%This is the second \CodeCommand{SubBulletItem}.
%\SubBulletItem
%The \CodeCommand{SubBulletItem}'s are between
%\CodeCommand{begin\{detail\}} and
%\CodeCommand{end\{detail\}} so that they are typeset in a smaller font.
%\end{detail}
%
%\Gap
%\BulletItem
%This is the third \CodeCommand{BulletItem}.
%
%\Gap
%\BulletItem
%A \CodeCommand{Gap} or \CodeCommand{GapNoBreak} is inserted between the \CodeCommand{BulletItem}'s so that there is a small vertical space between them.
%The ``NoBreak'' version prevents page breaking, and should be used to avoid orphaned headings and other formatting issues.
%
%\BigGap
%\subsection
%{This is the Second Subsection}
%{This is the Second Subsection}
%{PDF:ThisIsTheSecondSubSection}
%
%\GapNoBreak
%\BulletItem
%A \CodeCommand{BigGap} or \CodeCommand{BigGapNoBreak} is inserted between subsections so that there is a bigger vertical space between them.
%The ``NoBreak'' version prevents page breaking.
%
%%%%%%%%%%%%%%%%%%%%%%%%%%%%%%%%%%%%%%%%%%
%%% ANOTHER SECTION WITH USAGE EXAMPLES %%
%%%%%%%%%%%%%%%%%%%%%%%%%%%%%%%%%%%%%%%%%%
%
%\section
%{Another\newline
%Section\newline
%With\newline
%Usage\newline
%Examples}
%{Another Section With Usage Examples (For PDF Bookmark)}
%{PDF:AnotherSectionWithUsageExamples:ForPDFLink}
%
%\textbf{This is a Plain Heading},
%followed by an \CodeCommand{hfill} and a date range
%\hfill
%\DatestampYM{2015}{10} --
%\DatestampYM{2015}{12}
%
%\GapNoBreak
%\BulletItem
%This is a \CodeCommand{BulletItem}.
%\begin{detail}
%\SubBulletItem
%This is a \CodeCommand{SubBulletItem}.
%\end{detail}
%
%\GapNoBreak
%\BulletItem
%This is a \CodeCommand{BulletItem}.
%\begin{detail}
%\SubItem
%This is a \CodeCommand{SubItem}, which has no bullet.
%Note the alignment with the \CodeCommand{BulletItem} above.
%\end{detail}
%
%\GapNoBreak
%\Item
%This is an \CodeCommand{Item}, which has no bullet.
%Note the alignment with the \CodeCommand{BulletItem} above.
%\begin{detail}
%\SubItem
%This is a \CodeCommand{SubItem}.
%\end{detail}
%
%\GapNoBreak
%\NumberedItem{[16]}
%This is a \CodeCommand{NumberedItem}.
%Note the alignment with the \CodeCommand{SubBulletItem} above.
%
%\GapNoBreak
%\NumberedItem{{\CharSpace}[6]}
%This is a \CodeCommand{NumberedItem} with a \CodeCommand{CharSpace} in its argument for padding shorter numbers.
%Note the alignment with the \CodeCommand{NumberedItem} above.
%
%\BigGap
%\textbf{Usage Notes}
%
%\GapNoBreak
%\BulletItem
%New Lines and Paragraphs
%\begin{detail}
%\SubBulletItem
%To create a new line within the same paragraph (i.e., with the same indentation), use \CodeCommand{newline} instead of \CodeCommand{\textbackslash}.
%The latter will not work because it breaks the long table.
%\SubBulletItem
%To create a new paragraph, use \CodeCommand{par} or simply leave an empty line.
%Paragraph indentations (from
%\CodeCommand{Item},
%\CodeCommand{SubItem},
%\CodeCommand{BulletItem},
%\CodeCommand{SubBulletItem},
%etc.) do not carry across different paragraphs.
%\end{detail}
%
%\Gap
%\BulletItem
%Vertical Spacing Between Items
%\begin{detail}
%\SubBulletItem
%Use \CodeCommand{Gap} or \CodeCommand{GapNoBreak} to insert a small vertical space between items within the same section.
%\SubBulletItem
%Use \CodeCommand{BigGap} or \CodeCommand{BigGapNoBreak} to insert a bigger vertical space between items within the same section.
%\SubBulletItem
%The ``NoBreak'' versions prevent page breaking.
%\end{detail}
%
%\Gap
%\BulletItem
%Dates
%\begin{detail}
%\SubBulletItem
%Use
%\CodeCommand{DatestampYMD\{YYYY\}\{MM\}\{DD\}},
%\CodeCommand{DatestampYM\{YYYY\}\{MM\}}, and
%\CodeCommand{DatestampY\{YYYY\}}
%to specify dates.
%\SubBulletItem
%Change the date format option passed to the document class to adjust how dates are displayed throughout the document:
%MMMyyyy (``Dec~2010''),
%ddMMMyyyy (``31~Dec~2010''),
%MMMMyyyy (``December~2010''),
%ddMMMMyyyy (``31~December~2010''),
%yyyyMMdd (``2010-12-31''),
%yyyyMM (``2010-12''),
%yyyy (``2010'').
%\end{detail}
%
%%%%%%%%%%%%%%%%%%%%%%%%%%%%%%%%%%%%
%%% MULTILINGUAL UNICODE EXAMPLES %%
%%%%%%%%%%%%%%%%%%%%%%%%%%%%%%%%%%%%
%
%\section
%{Multilingual Unicode Examples}
%{Multilingual Unicode Examples}
%{PDF:MultilingualUnicodeExamples}
%
%\BulletItem
%Assortment of unicode characters from
%\href{http://www.ltg.ed.ac.uk/~richard/unicode-sample.html}
%{http://www.ltg.ed.ac.uk/{\TildeSymbol}richard/unicode-sample.html}
%
%\begin{detail}
%\SubItem
%\textbf{Latin Extended-A}
%Ā ā Ă ă Ą ą Ć ć Ĉ ĉ Ċ ċ Č č Ď ď Đ đ Ē ē Ĕ ĕ Ė ė Ę ę Ě ě Ĝ ĝ Ğ ğ Ġ ġ Ģ ģ Ĥ ĥ Ħ ħ Ĩ ĩ Ī ī Ĭ ĭ Į į İ ı IJ ij Ĵ ĵ
%\textbf{Latin Extended-B}
%ƀ Ɓ Ƃ ƃ Ƅ ƅ Ɔ Ƈ ƈ Ɖ Ɗ Ƌ ƌ ƍ Ǝ Ə Ɛ Ƒ ƒ Ɠ Ɣ ƕ Ɩ Ɨ Ƙ ƙ ƚ ƛ Ɯ Ɲ ƞ Ɵ Ơ ơ Ƣ ƣ Ƥ ƥ Ʀ Ƨ ƨ Ʃ ƪ ƫ Ƭ ƭ Ʈ Ư ư Ʊ Ʋ Ƴ ƴ Ƶ
%\textbf{Latin Extended Additional}
%Ḁ ḁ Ḃ ḃ Ḅ ḅ Ḇ ḇ Ḉ ḉ Ḋ ḋ Ḍ ḍ Ḏ ḏ Ḑ ḑ Ḓ ḓ Ḕ ḕ Ḗ ḗ Ḙ ḙ Ḛ ḛ Ḝ ḝ Ḟ ḟ Ḡ ḡ Ḣ ḣ Ḥ ḥ Ḧ ḧ Ḩ ḩ Ḫ ḫ Ḭ ḭ Ḯ ḯ Ḱ ḱ Ḳ ḳ Ḵ ḵ
%\textbf{Greek}
%ʹ ͵ ͺ ; ΄ ΅ Ά · Έ Ή Ί Ό Ύ Ώ ΐ Α Β Γ Δ Ε Ζ Η Θ Ι Κ Λ Μ Ν Ξ Ο Π Ρ Σ Τ Υ Φ Χ Ψ Ω Ϊ Ϋ ά έ ή ί ΰ α β γ δ ε ζ η θ
%\textbf{Cyrillic}
%Ё Ђ Ѓ Є Ѕ І Ї Ј Љ Њ Ћ Ќ Ў Џ А Б В Г Д Е Ж З И Й К Л М Н О П Р С Т У Ф Х Ц Ч Ш Щ Ъ Ы Ь Э Ю Я а б в г д е ж з
%\textbf{Hebrew}
%א ב ג ד ה ו ז ח ט י ך כ ל ם מ ן נ ס ע ף פ ץ צ ק ר ש ת װ ױ ײ ֝ ֞ ֟ ֠ ֡ ֣ ֤ ֥ ֦ ֧ ֨ ֩ ֪ ֫ ֬ ֭ ֮ ֯ ְ ֱ ֒ ֓ ֔
%\textbf{Armenian}
%{\UseSecondaryFont
%Ա Բ Գ Դ Ե Զ Է Ը Թ Ժ Ի Լ Խ Ծ Կ Հ Ձ Ղ Ճ Մ Յ Ն Շ Ո Չ Պ Ջ Ռ Ս Վ Տ Ր Ց Ւ Փ Ք Օ Ֆ ՙ ՚ ՛ ՜ ՝ ՞ ՟ ա բ գ դ ե զ}
%\textbf{Thai}
%{\UseSecondaryFont
%ก ข ฃ ค ฅ ฆ ง จ ฉ ช ซ ฌ ญ ฎ ฏ ฐ ฑ ฒ ณ ด ต ถ ท ธ น บ ป ผ ฝ พ ฟ ภ ม ย ร ฤ ล ฦ ว ศ ษ ส ห ฬ อ ฮ ฯ ะ ั า ำ ิ}
%\end{detail}
%
\end{body}

%%%%%%%%%%%
% CV NOTE %
%%%%%%%%%%%

\UseNoteFont%
\null\hfill%
\textit{\CVNote}%
\hspace{2.0mm}\null

\end{document}
